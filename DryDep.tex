%% Copernicus Publications Manuscript Preparation Template for LaTeX Submissions
%% ---------------------------------
%% This template should be used for copernicus.cls
%% The class file and some style files are bundled in the Copernicus Latex Package, which can be downloaded from the different journal webpages.
%% For further assistance please contact Copernicus Publications at: production@copernicus.org
%% https://publications.copernicus.org/for_authors/manuscript_preparation.html


%% Please use the following documentclass and journal abbreviations for discussion papers and final revised papers.

%% 2-column papers and discussion papers
\documentclass[gmd, manuscript]{copernicus}



%% Journal abbreviations (please use the same for discussion papers and final revised papers)


% Advances in Geosciences (adgeo)
% Advances in Radio Science (ars)
% Advances in Science and Research (asr)
% Advances in Statistical Climatology, Meteorology and Oceanography (ascmo)
% Annales Geophysicae (angeo)
% Archives Animal Breeding (aab)
% ASTRA Proceedings (ap)
% Atmospheric Chemistry and Physics (acp)
% Atmospheric Measurement Techniques (amt)
% Biogeosciences (bg)
% Climate of the Past (cp)
% DEUQUA Special Publications (deuquasp)
% Drinking Water Engineering and Science (dwes)
% Earth Surface Dynamics (esurf)
% Earth System Dynamics (esd)
% Earth System Science Data (essd)
% E&G Quaternary Science Journal (egqsj)
% Fossil Record (fr)
% Geographica Helvetica (gh)
% Geoscientific Instrumentation, Methods and Data Systems (gi)
% Geoscientific Model Development (gmd)
% History of Geo- and Space Sciences (hgss)
% Hydrology and Earth System Sciences (hess)
% Journal of Micropalaeontology (jm)
% Journal of Sensors and Sensor Systems (jsss)
% Mechanical Sciences (ms)
% Natural Hazards and Earth System Sciences (nhess)
% Nonlinear Processes in Geophysics (npg)
% Ocean Science (os)
% Primate Biology (pb)
% Proceedings of the International Association of Hydrological Sciences (piahs)
% Scientific Drilling (sd)
% SOIL (soil)
% Solid Earth (se)
% The Cryosphere (tc)
% Web Ecology (we)
% Wind Energy Science (wes)


%% \usepackage commands included in the copernicus.cls:
%\usepackage[german, english]{babel}
%\usepackage{tabularx}
%\usepackage{cancel}
%\usepackage{multirow}
%\usepackage{supertabular}
%\usepackage{algorithmic}
%\usepackage{algorithm}
%\usepackage{amsthm}
%\usepackage{float}
%\usepackage{subfig}
%\usepackage{rotating}


\begin{document}

\title{Update of the ozone dry deposition in the OsloCTM3 v1.0}


% \Author[affil]{given_name}{surname}

\Author[1]{Stefanie}{Falk}
\Author[2,a]{Amund}{S{\o}vde Haslerud}
%\Author[]{}{}

\affil[1]{Department of Geosciences, University of Oslo, Oslo, Norway}
\affil[2]{CICERO Center for International Climate Research, Oslo, Norway}
\affil[a]{Kjeller Vindteknikk, Kjeller, Norway}

%% The [] brackets identify the author with the corresponding affiliation. 1, 2, 3, etc. should be inserted.



\runningtitle{Update of ozone dry deposition in OsloCTM3}

\runningauthor{Falk}

\correspondence{Stefanie Falk (stefanie.falk@geo.uio.no)}



\received{}
\pubdiscuss{} %% only important for two-stage journals
\revised{}
\accepted{}
\published{}

%% These dates will be inserted by Copernicus Publications during the typesetting process.


\firstpage{1}

\maketitle



\begin{abstract}
  Since the industrial revolution, tropospheric background ozone concentrations have been increasing strongly mainly in the northern hemisphere. In recent years, the number of episodes of peak concentrations especially in summertime has been decreasing in North America and Europe due to the implementation of air quality regulations. At the same time, fast developing countries, like e.g., China or India, saw a significant increase in ozone related air pollution. High concentrations of ozone in ambient air are hazardous not only to humans but to the ecosystem in general. The impact of ozone damage on vegetation and agricultural plants in particular may even affect food security in the future. Most chemistry-transport models (CTMs) and chemistry-climate models (CCMs) tend to overestimate surface ozone values. This may be partly due to uncertainties in the distribution and amount of precursor species and stratosphere-troposphere exchange as well as scavenging processes removing ozone from the atmosphere. Removal of any gas through gravitational settling or by uptake by plants and soil is referred to as dry deposition. The process of dry deposition is important for predicting surface ozone concentrations and understanding the observed amount and increase of tropospheric background ozone. Dry deposition velocities are typically calculated following a resistance-analogous approach wherein aerodynamic, quasi laminar, and canopy resistances are key components, but these conceptual dry deposition velocities are hard to measure explicitly. Following the European Monitoring and Evaluation Programme (EMEP) MSC-W chemistry transport model, we compute canopy resistance from stomatal conductance. In this paper, we present an update on the dry deposition scheme implemented in the OsloCTM3v1.0 and evaluate the model with respect to ozone and different parameters in the stomatal conductance parameterization, e.g., photosynthetic photon flux density (PPFD) and phenology. Furthermore, we investigate the impact on modeled ozone concentrations, both at the surface and in the free troposphere. We show that ozone dry deposition in the new version (v.1.1) decreases with respect to the previous version (v1.0), leading to an increase in surface ozone of up to 40\,\unit{\%}. We also put our results in context to recent multi-model studies as well as satellite data.
\end{abstract}


%\copyrightstatement{TEXT}
\introduction  %% \introduction[modified heading if necessary]
Ozone is an important trace gas for all lifeforms on Earth. Depending on the place of its occurrence it has either a positive or negative connotation. In the stratosphere, ozone absorbs most of the ultraviolet (UV)-light from the sun within the range of 100--315\,\unit{nm}, thus shielding the Earth's surface from the most harmful UV-radiation. In addition, ozone is a potent greenhouse gas in both, stratosphere and troposphere. With a radiative forcing of $0.40 \pm 0.20\,\unit{Wm^{-2}}$, it is placed third, only surpassed by \chem{CO_2} and \chem{CH_4} \citep[Chapter 8]{IPCC2013}.\\
In the troposphere and in particular in ambient air, ozone is considered as a highly toxic pollutant. Continuously high concentrations of ambient air ozone are hazardous to the whole ecosystem. It is estimated that ozone is cause to an increase in pre-mature deaths \citep{WHO2008}, an average global loss of yield in the four major crops (wheat, rice, maize, and soybean) of about 3--15\,\unit{\%} \citep{PJ:Ainsworth2017} as well as 7\,\unit{\%} loss in primary production in forestry \citep{GCB:Wittig2009,EP:Matyssek2012}. The impact of ozone damage on vegetation and agricultural plants in particular may even affect food security in the future especially in Asia \citep{GCB:Tang2013,NCC:Tai2014,AE:Chuwah2015} and might be an important additional feedback to climate change \citep{Nat:Sitch2007}.\\
Tropospheric ozone is mainly produced in situ in complex photochemical cycles involving precursor gases such as carbon monoxide (\chem{CO}) or volatile organic substances (VOCs) and hydrocarbons (e.g., \chem{CH_4}) in the presents of nitrogen oxides (\chem{NO_x}). A typical reaction cylce for \chem{CO} is sketched below. In a sequence of rapid reactions a peroxy radical \chem{HO_2^\bullet} is formed through an initial reaction of \chem{CO} with a hydroxyl radical \chem{^\bullet OH}
\begin{reaction}
  ^\bullet\chem{OH} + \chem{CO} \rightarrow {^\bullet\chem{HOCO}}\\
  ^\bullet\chem{HOCO} + \chem{O_2} \rightarrow \chem{HO_2^\bullet} + \chem{CO_2}.
\end{reaction}
Via a reaction between \chem{HO_2^\bullet} and \chem{NO}, \chem{NO_2} is formed which is then photolysed. The resulting atomic oxygen reacts then with \chem{O_2} to form an ozone molecule
\begin{reaction}
  \chem{HO_2^\bullet} + \chem{NO} \rightarrow {^\bullet\chem{OH}} + \chem{NO_2}\\
  \chem{NO_2} + h\nu \rightarrow \chem{NO} + \chem{O(^3{P})}\\
  ^\bullet\chem{O(^3P)} + \chem{O_2} \rightarrow \chem{O_3}.
\end{reaction}
Such a cycle leads to a net production via:
\begin{reaction}
  \chem{CO} + 2\chem{O_2} + h\nu \rightarrow \chem{CO_2} + \chem{O_3}.
\end{reaction}
A part from this, additional reaction pathways involving volatile organic compounds (VOCs) exist. Another possible source of tropospheric ozone is downward transport from the stratosphere via stratosphere-troposphere exchange (STE) \citep{WMO2014}. Major sinks are photolysis to \chem{O({^1{D}})} followed by a reaction with water vapor, reaction with \chem{HO_2} \citep{ACP:Seinfeld2006}, and de-nitration via
\begin{reaction}
  \chem{NO} + \chem{O_3} \rightarrow \chem{NO_2} + \chem{O_2}
\end{reaction}
and
\begin{reaction}
  \chem{NO_2} + \chem{O_3} \rightarrow \chem{NO_3} + \chem{O_2},
\end{reaction}
and scavenging processes, in particular dry deposition. A very special sink which is linked to dry deposition are so called ozone depleting events. They occure in spring-time in the polar boundary layer where an outburst of bromine monoxid \chem{BrO} (so called bromine explosion) leads to a rapid depletion of surface ozone \citep{JGR:Oltmans1981,GRL:Bottenheim1986,Nat:Barrie1988,JGR:Bottenheim2006}. Various schemes have been sucessfully applied to different types of atmospheric models \citep[e.g.][]{ACP:Yang2010,ACP:Toyota2011,ACP:Toyota2014a,GMD:Falk2018}. Although these events are important to understand surface ozone abundance in Arctic regions, we have not implemented any parameterization of these processes, yet.\\
Since ozone is highly reactive, its global mean life-time in the troposphere is roughly 22 days but ranges between a few days in the tropical boundary layer to up to one year in the upper troposphere \citep{JGR:Stevenson2005,ACP:Young2013}. The abundance of tropospheric ozone therefore varies, e.g., with time of the day (maximum $\sim$15:00 local time), season (mid-June maximum), altitude, and location \citep{ACP:Schnell2015} or weather conditions in general \citep{ACP:Otero2018}. Typical concentrations of surface ozone range from 10\,\unit{ppb} over the tropical Pacific to 100\,\unit{ppb} in the downwind areas of highly emitting sources \citep[Chapter 8]{IPCC2013}. This variability poses a challenge on both, trend analysis from observation as well as validation and intercomparison of models. At the observational side, there is only a limited number of long-term ozone observations, mainly restricted to European sites. Amongst these, only three display a statistical significant trend, indicating a doubling of tropospheric ozone since the onset of the industrialization \citep[Chapter 2]{IPCC2013}. But especially the very low pre-industrial ozone abundance cannot be reproduced by the likes of most models, while there is a general tendency to underestimate tropospheric ozone among the participating models in the Atmospheric Chemistry and Climate Model Intercomparison Project (ACCMIP) \citep[Chapter 8]{IPCC2013}.\\
Removal of any gas through gravitational settling or by uptake by plants and soil is referred to as dry deposition. The process of dry deposition is important for predicting surface ozone concentrations and understanding the observed amount and increase of tropospheric background ozone. Dry deposition velocities $v^i_d$ (also refered to as contuctances $G^i$) for any gaseous specie $i$ are typically calculated following a resistance-analogous approach
\begin{equation}
  v^i_d = \frac{1}{R_a + R^i_b + R^i_c},
\end{equation}
wherein aerodynamic $R_a$, quasi laminar $R^i_b$, and canopy resistances $R^i_c$ are key components \citep{AE:Wesely1989}. For all gases $R_a$ is the same, while $R^i_b$ and $R^i_c$ vary from gaseous specie to specie and also depend on landuse types (e.g., ice/snow, water, urban, desert, agricultural land, deciduous forest, coniferous forest etc.). In its original formulation, fixed seasonal average dry deposition resistances have been used for each landuse type. For all three types of resistances in this Wesely-type parameterization, more process-oriented formulations have been developed and validated over the years, e.g., dry deposition of ozone to ocean \citep{ACP:Luhar2017} and stomatal conductance as part of canopy conductance \citep{PP:Buckley2017} or updates on the parameters have been provided from model validation with observation \citep[e.g., $v^\chem{O_3}_{ice/snow} = 1/10000\,\unit{cm s^{-1}}$,][]{ACP:Helmig2007}. 




What is the status and what are the challenges in the dry deposition of ozone.


Impact of ocean? Although it is a low number there is 2\/3 ocean on earth... (Think, I read this in Hardacre paper... -> Luhar)




\section{Model description}
General model description followed by detailed dry deposition scheme \citep{ACP:Simpson2012,ICP:MappingManual2017}. This is followed by the necessary pre-processing of files. Old scheme \citep{AE:Wesely1989, JGR:Hough1991}

historical anthropogenic and biomass burning emissions \citep{Lamarque2010}, meteorological driver \href{https://www.ecmwf.int/en/forecasts/documentation-and-support/evolution-ifs/cycle-38r1-summary-changes}{ECMWF - OpenIFS version cy38r1nc4}, Community Emission Data System (CEDS) historical emission inventory \citep{GMD:Hoesly2018}

\citet{GMD:Sovde2012,GMD:Lund2018}
\subsection{Ozone dry deposition scheme}
\label{subsec:DryDep}
Convention on Long-Range Transboundary Air Pollution (CLRTAP) -> European Monitoring and Evaluation Programme (EMEP),
simple plant function type (PFT)-based scheme. In this scheme, an average canopy resistance is computed from stomatal conductance of fractional PFTs in a grid cell.
\citep{ACP:Simpson2012} \citet{ICP:MappingManual2017}
\subsection{Greening season and photoactive radiation}

\section{Evaluation}

\subsection{Sensitivity tests}
Include table of all sensitivity runs.
\citep[e.g., $v^\chem{O_3}_{ice/snow} = 1/10000\,\unit{cm s^{-1}}$,][]{ACP:Helmig2007}
dry dep velocity over water \citep{JGR:Helmig2012}

\subsection{Comparison with recent multi-model evaluation}
Compare to \citet{ACP:Hardacre2015} as well as \citet{ACP:Luhar2017}.

\subsection{Comparison with observations}
The sites used in \citet{ACP:Hardacre2015} and additional ozone zonal data from satellites and sondes (\url{http://bodekerscientific.com/}).

\section{Discussion}


%\subsection{HEADING}
%\subsubsection{HEADING}



\conclusions  %% \conclusions[modified heading if necessary]
What did we gain?

%% The following commands are for the statements about the availability of data sets and/or software code corresponding to the manuscript.
%% It is strongly recommended to make use of these sections in case data sets and/or software code have been part of your research the article is based on.

%\codeavailability{TEXT} %% use this section when having only software code available


%\dataavailability{TEXT} %% use this section when having only data sets available


\codedataavailability{TEXT} %% use this section when having data sets and software code available


%\sampleavailability{TEXT} %% use this section when having geoscientific samples available



%\appendix
%\section{}    %% Appendix A

%\subsection{}     %% Appendix A1, A2, etc.


%\noappendix       %% use this to mark the end of the appendix section

%% Regarding figures and tables in appendices, the following two options are possible depending on your general handling of figures and tables in the manuscript environment:

%% Option 1: If you sorted all figures and tables into the sections of the text, please also sort the appendix figures and appendix tables into the respective appendix sections.
%% They will be correctly named automatically.

%% Option 2: If you put all figures after the reference list, please insert appendix tables and figures after the normal tables and figures.
%% To rename them correctly to A1, A2, etc., please add the following commands in front of them:

%\appendixfigures  %% needs to be added in front of appendix figures

%\appendixtables   %% needs to be added in front of appendix tables

%% Please add \clearpage between each table and/or figure. Further guidelines on figures and tables can be found below.



\authorcontribution{Stefanie Falk wrote the paper, did the data analysis, and finalized the implementation of the stomatal conductance in the dry deposition scheme. Amund S{\o}vde Haslerud did most of the implementation and documentation of the updated dry deposition scheme.} %% it is strongly recommended to make use of this section

\competinginterests{The authors declare that they have no conflict of interest.} %% this section is mandatory even if you declare that no competing interests are present

%\disclaimer{TEXT} %% optional section

\begin{acknowledgements}
  This work was supported by the Norwegian Research Council (NRC) through the project The double punch: Ozone and climate stresses on vegetation (OzoNorClim).\\
  We would like to thank Greg Bodeker, Stefanie Kremser (Bodeker Scientific) and Birgit Hassler (DLR) for providing the combined vertical ozone profile database (\url{http://www.bodekerscientific.com}.
  The used Leaf Area Index (LAI) and roughness length ($Z_0$) are available online from Oak Ridge National Laboratory Distributed Active Archive Center, Oak Ridge, Tennessee, U.S.A. (\doi{10.3334/ORNLDAAC/970}).\\
  Community Emission Data System (CEDS) historical emission inventory is provided by the Joint Global Research Institute project (\url{http://www.globalchange.umd.edu/ceds/}.)
\end{acknowledgements}




%% REFERENCES

%% The reference list is compiled as follows:

%% Since the Copernicus LaTeX package includes the BibTeX style file copernicus.bst,
%% authors experienced with BibTeX only have to include the following two lines:
%%
\bibliographystyle{copernicus}
\bibliography{DryDep.bib}
%%
%% URLs and DOIs can be entered in your BibTeX file as:
%%
%% URL = {http://www.xyz.org/~jones/idx_g.htm}
%% DOI = {10.5194/xyz}


%% LITERATURE CITATIONS
%%
%% command                        & example result
%% \citet{jones90}|               & Jones et al. (1990)
%% \citep{jones90}|               & (Jones et al., 1990)
%% \citep{jones90,jones93}|       & (Jones et al., 1990, 1993)
%% \citep[p.~32]{jones90}|        & (Jones et al., 1990, p.~32)
%% \citep[e.g.,][]{jones90}|      & (e.g., Jones et al., 1990)
%% \citep[e.g.,][p.~32]{jones90}| & (e.g., Jones et al., 1990, p.~32)
%% \citeauthor{jones90}|          & Jones et al.
%% \citeyear{jones90}|            & 1990



%% FIGURES

%% When figures and tables are placed at the end of the MS (article in one-column style), please add \clearpage
%% between bibliography and first table and/or figure as well as between each table and/or figure.


%% ONE-COLUMN FIGURES

%%f
%\begin{figure}[t]
%\includegraphics[width=8.3cm]{FILE NAME}
%\caption{TEXT}
%\end{figure}
%
%%% TWO-COLUMN FIGURES
%
%%f
%\begin{figure*}[t]
%\includegraphics[width=12cm]{FILE NAME}
%\caption{TEXT}
%\end{figure*}
%
%
%%% TABLES
%%%
%%% The different columns must be seperated with a & command and should
%%% end with \\ to identify the column brake.
%
%%% ONE-COLUMN TABLE
%
%%t
%\begin{table}[t]
%\caption{TEXT}
%\begin{tabular}{column = lcr}
%\tophline
%
%\middlehline
%
%\bottomhline
%\end{tabular}
%\belowtable{} % Table Footnotes
%\end{table}
%
%%% TWO-COLUMN TABLE
%
%%t
%\begin{table*}[t]
%\caption{TEXT}
%\begin{tabular}{column = lcr}
%\tophline
%
%\middlehline
%
%\bottomhline
%\end{tabular}
%\belowtable{} % Table Footnotes
%\end{table*}
%
%%% LANDSCAPE TABLE
%
%%t
%\begin{sidewaystable*}[t]
%\caption{TEXT}
%\begin{tabular}{column = lcr}
%\tophline
%
%\middlehline
%
%\bottomhline
%\end{tabular}
%\belowtable{} % Table Footnotes
%\end{sidewaystable*}
%
%
%%% MATHEMATICAL EXPRESSIONS
%
%%% All papers typeset by Copernicus Publications follow the math typesetting regulations
%%% given by the IUPAC Green Book (IUPAC: Quantities, Units and Symbols in Physical Chemistry,
%%% 2nd Edn., Blackwell Science, available at: http://old.iupac.org/publications/books/gbook/green_book_2ed.pdf, 1993).
%%%
%%% Physical quantities/variables are typeset in italic font (t for time, T for Temperature)
%%% Indices which are not defined are typeset in italic font (x, y, z, a, b, c)
%%% Items/objects which are defined are typeset in roman font (Car A, Car B)
%%% Descriptions/specifications which are defined by itself are typeset in roman font (abs, rel, ref, tot, net, ice)
%%% Abbreviations from 2 letters are typeset in roman font (RH, LAI)
%%% Vectors are identified in bold italic font using \vec{x}
%%% Matrices are identified in bold roman font
%%% Multiplication signs are typeset using the LaTeX commands \times (for vector products, grids, and exponential notations) or \cdot
%%% The character * should not be applied as mutliplication sign
%
%
%%% EQUATIONS
%
%%% Single-row equation
%
%\begin{equation}
%
%\end{equation}
%
%%% Multiline equation
%
%\begin{align}
%& 3 + 5 = 8\\
%& 3 + 5 = 8\\
%& 3 + 5 = 8
%\end{align}
%
%
%%% MATRICES
%
%\begin{matrix}
%x & y & z\\
%x & y & z\\
%x & y & z\\
%\end{matrix}
%
%
%%% ALGORITHM
%
%\begin{algorithm}
%\caption{...}
%\label{a1}
%\begin{algorithmic}
%...
%\end{algorithmic}
%\end{algorithm}
%
%
%%% CHEMICAL FORMULAS AND REACTIONS
%
%%% For formulas embedded in the text, please use \chem{}
%
%%% The reaction environment creates labels including the letter R, i.e. (R1), (R2), etc.
%
%\begin{reaction}
%%% \rightarrow should be used for normal (one-way) chemical reactions
%%% \rightleftharpoons should be used for equilibria
%%% \leftrightarrow should be used for resonance structures
%\end{reaction}
%
%
%%% PHYSICAL UNITS
%%%
%%% Please use \unit{} and apply the exponential notation


\end{document}
